\documentclass[11pt]{amsart}
\usepackage{geometry}                % See geometry.pdf to learn the layout options. There are lots.
\geometry{letterpaper}                   % ... or a4paper or a5paper or ... 
%\geometry{landscape}                % Activate for for rotated page geometry
%\usepackage[parfill]{parskip}    % Activate to begin paragraphs with an empty line rather than an indent
\usepackage{graphicx}
\usepackage{amssymb}
\usepackage{epstopdf}
\DeclareGraphicsRule{.tif}{png}{.png}{`convert #1 `dirname #1`/`basename #1 .tif`.png}
\usepackage{amsfonts,amsmath,amsthm,amsbsy,latexsym,amssymb,graphicx,booktabs,multicol,color,fullpage}
\usepackage{mathabx,dashrule}
\usepackage{manfnt}
\usepackage{pgf,tikz}
\usepackage{pgfplots}
\usepackage{graphicx}
\usepackage{wrapfig}
\usepackage{multicol}

\title{Thesis Proposal}
\author{Melanie Vining}
%\date{}                                           % Activate to display a given date or no date

\begin{document}
\maketitle
\section{Introduction} \\
\subsection{Fourier Continuation} \\
Fourier Continuation (FC) is an approximation method used to extend the computational abilities of a Fourier Series to non-periodic functions.  
\subsection{Past Work}
\section{Current Work}
\subsection{Application to the Heat Equation} \\
We study the Boundary Value Problem $(I-\alpha \frac{\partial^2}{\partial x^2})u=f$ with boundary values $u(a)=u_0$ and $u(b)=u_1$.  As a motivating example, consider $f(x)=x$ on $[0,1]$.  Let $\mathcal{L}=(I-\alpha \frac{\partial^2}{\partial x^2}$.  When we solve $u=\mathcal{L}^{-1}f$ using Fourier Continuation approximations, some energy from the continuation domain can be spread back into the system.  Since we know analytically for $\alpha >0$ the system is stable, using the Fourier Continuation approximation can be a computationally inaccurate approach. Our goal is to find an additional constraint that would preserve the accuracy of the Fourier Continuation approximation while ensuring the stability of the operation.  
\subsection{Green's Functions} \\
\subsection{Results} \\
For the individual Gram Polynomials, we saw accuracy of $\mathcal{O}(10^{-14})$.  In the following figure, we see the accuracy as a function of parameter $\alpha$
\section{Future Work}
\subsection{Computational Work} \\
We are going to use this to put as a time step of the heat equation and solve that PDE.  Our goal is to show that we have a stable approximation that can be used. 
\subsection{Analytical Work} \\
The result that yields the same Fourier coefficients for any given Gram polynomial independent of choice of $\alpha$ is unexpected.  Our goal is to develop an analytic proof that justifies this result in general. 



\end{document}  